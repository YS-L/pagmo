
\title{Features for the GSoC 2010 Project:\\
       3D Visualization for Interplanetary Trajectories}
\author{
      Edgar Sim\'{o} Serra
}
\date{\today}

\documentclass[12pt]{article}

\begin{document}
\maketitle

\begin{abstract}
Future intended feature set for the 3D Visualization for Interplanetary Trajectories GSoC 2010 project.
\end{abstract}

\section{Introduction}
The primary goal of this project is proper and flexible visualization of interplanetary trajectories, therefore this functionality should be the prioritized over other niceties. However that does not mean other features should not be implemented, they will be done as time allows.

\section{Features}

Features are split into two main groups: Core features and Important features. The core features are features which are a fundamental part of the project and that without them the project wouldn't be useful. Important features are features that while being very important for the usefulness of the project they aren't fundamental to it.

\paragraph{Core features:}
\begin{itemize}
\item Visualization of trajectories
\item Ability to see velocity and trajectory vectors
\item Time slider to be able to move to any time instant in the path
\item Keyboard interaction
\item Mouse Interaction that behaves like most 3D applications
\item Sane reference system to allow to allow changing system coordinates
\end{itemize}

\paragraph{Important features:}
\begin{itemize}
\item Flexible auto-camera (smooth following)
\item Flexible auto-reference (jump to nearest object)
\item Save as SVG
\item Annotation system
\item Save videos (between intervals)
\item Allow changing of models
\end{itemize}


\end{document}

